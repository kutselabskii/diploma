\documentclass{article}
\usepackage{polyglossia}
\setmainfont{Times New Roman}

\begin{document}
    \section{Титульник}
        \par Защищается студент группы Б8403а Куцелабский Егор Сергеевич по теме «Текстовый 
        процессор для open-source движка Citrus»
        \par Руководитель – старший преподаватель кафедры информатики, математического и 
        компьютерного моделирования, Кленин Александр Сергеевич
    \section{Game Forest и текст}
        \par Студия Game Forest занимается созданием игр. Текст занимает важную часть в 
        информационных сообщениях в играх. Так же текст используется в визуальных редакторах 
        игрового движка, разрабатываемого компанией. Текущая реализация текстового процессора в 
        движке не обладает достаточной эффективностью, поэтому, при увеличении количества 
        текста, изображаемого на экране, частота кадров в приложениях падает. 
    \section{Текстовые процессоры}
        \par Для  редактирования текстовых данных существуют текстовые процессоры.
        \par Они хранят текст, используя при этом различные по эффективности структуры данных, 
        вычисляют его размеры, отрисовывают, некоторые редакторы (например, Word), позволяют, 
        помимо редактирования текста, использовать в документе изображения, таблицы и прочие не 
        относящиеся непосредственно к тексту элементы.
    \section{Цели и задачи работы}
        \par Цель работы – создание нового текстового процессора для движка.
        \par Для достижения цели были поставлены следующие задачи:
        \begin{itemize}
            \item Изучить эффективные методы представления текста
            \item Реализовать текстовый процессор
            \item Оценить выигрыш в производительности
        \end{itemize}
        \textbf{А МОЖЕТ ЦЕЛИ ПРИВЕДЕНЫ НА СЛАЙДЕ?}
    \section{Структуры данных}
        \par Мною были рассмотрены различные подходы к реализации текстовых редакторов. 
        Сравнительную таблицу алгоритмических сложностей подходов для базовых операций вы 
        можете увидеть на экране. Стоит отметить, что она не вполне отражает реальные 
        характеристики подходов, поскольку типичная производительность подходов значительно 
        отличается от производительности в худшем случае и её сложно верно оценить. При 
        наполнении таблицы использовались данные для худших возможных случаев для 
        соответствующих подходов. Подробную информацию о перечисленных в таблице подходах вы 
        можете прочитать в отчете.
    \section{Функциональные требования}
        \begin{itemize}
            \item Эффективная обработка внутреннего представления текста ---
            Текст должен храниться в эффективной структуре данных, все базовые операции должны 
            выполняться быстро, в том числе для больших объемов текстовых данных (порядка 
            нескольких мегабайт) при этом структура не должна требовать значительных объемов 
            оперативной памяти.
            \item Отрисовка текста с заданными параметрами --- Текст должен отрисовываться в 
            окне заданного размера, с заданными параметрами шрифта, при этом отрисовываться 
            должна только та часть текста, которая попадает в видимую область окна.
            \item Обработка пользовательского ввода --- Процессор должен обрабатывать 
            пользовательский ввод – ввод текстовых данных, перемещение курсора с помощью мыши и 
            клавиш клавиатуры или касаний экрана выполняя при этом соответствующие операции по 
            редактированию текста.
        \end{itemize}
    \section{Существующая версия архитектуры}
        \par На слайде представлена диаграмма классов предыдущей версии текстового процессора.
        \par Эта реализация была признана неудачной. При оптимизации этой версии процессора 
        пришлось бы совершить двойную работу, переписывая похожие участки кода в двух классах, 
        в то время как функциональность RichText и SimpleText можно объединить в одном классе.
    \section{Архитектура системы: представление текста}
        \par Система состоит из трёх модулей.
        \par Модуль, отвечающий за обработку внутреннего представления текста – обработка 
        базовых команд, выделение, перемещение курсора, undo/redo.
        \par На слайде изображена диаграмма классов данного модуля.
    \section{Архитектура системы: отрисовка и ввод}
        \par Следующий модуль отвечает за отрисовку. В его основе существовавший ранее код, 
        однако он был значительно модифицирован для оптимизации (т. е. для отрисовки только 
        видимой части текста).
        \par Третий модуль отвечает за обработку пользовательского ввода, он принимает 
        поддерживаемые команды и передаёт их соответствующим обработчикам.
        \par На слайде изображены диаграммы классов перечисленных систем.
    \section{Piece Table}
        \par В качестве подхода для реализации был выбран метод Piece Table. Суть Piece Table 
        состоит в том, что вместо хранения символов в памяти хранится информация о позиции 
        текстового фрагмента в потоке, его длине, а также о том, к какому именно потоку 
        относится фрагмент. В Piece Table используется два потока: первый – исходный файл, этот 
        поток используется только для чтения. Второй – добавочный файл, весь новый текст 
        записывается в него.
        \par При необходимости вывести текст на экран выполняется пробег по структуре.
        \par На слайде представлен пример работы Piece Table.
    \section{Splay Tree}
        \par Piece table это структура для эффективного редактирования текста, необходима также 
        структура и для навигации по тексту.
        \par Splay tree – сбалансированное бинарное дерево поиска, основное свойство которого – 
        элементы, к которым обращались в последний раз, переносятся в корень дерева. Это 
        свойство очень полезно при работе с текстовыми редакторами, поскольку большая часть 
        операций вставки и удаления производится в позиции курсора, а частота её смены на 
        значительную величину невелика по сравнению с частотой прочих операций.
        \par Алгоритм балансировки с переносом элемента в корень имеет логарифмическую 
        амортизированную сложность. После переноса необходимого элемента в корень, любая 
        операция вставки, удаления или обращения к элементу выполняется за O(1). Подробнее об 
        алгоритме балансировки можно прочесть в отчете, там же приведена ссылка на статью с 
        доказательством сложности алгоритма.
        \par На слайде приведён пример операции переноса элемента в корень.
    \section{Интерфейс}
        \par На слайде представлены примеры интерфейсов. Слева вы можете видеть экземпляр класса
        Editor в интерфейсе движка Citrus, справа -- пример декорированного текстового виджета.
    \section{Тестирование: отрисовка}
        \par На слайде представлено сравнение времени отрисовки одного кадра в новой и старой
        версиях редактора при ограниченном числе строк. Максимальное количество строк, для
        которого проводилось сравнение - 50 тысяч, однако график с этим сравнением не 
        информативен, поскольку разница во времени между новой и старой версиями процессора 
        чрезвычайно велика. Скорость отрисовки одного кадра в старой версии процессора --
        примерно 1.5 секунды, в то время как скорость отрисовки в новой версии достигает своего
        минимума на сотне строк и после этого не изменяется. График этого сравнения вы можете
        посмотреть в отчёте.
    \section{Тестирование: вставка}
        \par На экране представлено сравнение времени вставки символа в редактор, уже 
        содержащий некоторое количество строк. Хотя выигрыш в скорости не столь значителен, как
        при отрисовке, стоит отметить, что во время вставки в новую версию процессора также
        происходит и пересчёт позиции курсора, в то время как в старой версии процессора на 
        этот подсчёт требовалось выделение дополнительного времени.
    \section{Реализация}
        \par Физические характеристики системы приведены на слайде
    \section{Заключение}
        \begin{itemize}
            \item Выполнен анализ структур данных для работы с текстом
            \item Реализована новая версия текстового процессора
            \item Реализованные модули показывают значительное улучшение производительности по 
            сравнению с предыдущей версией процессора
            \item В настоящий момент код находится на этапе опытной эксплуатации
        \end{itemize}
    \section{СПАСИБО ЗА ВНИМАНИЕ}
\end{document}