\documentclass{fefu}

\setschool{ШКОЛА ЕСТЕСТВЕННЫХ НАУК}
\setdepartment{кафедра информатики, математического и компьютерного моделирования}
\setdepartmenthead{А.Ю.Чеботарев}
\setgroup{Б8403а}
\title{о прохождении преддипломной практики\\направление подготовки 01.03.02 Прикладная математика и информатика\\профиль «Системное программирование»}
\setdates{29}{апреля}{2019}{22}{июня}{2019}
\setweeks{8}
\setplace{кафедре алгебры, геометрии и анализа}
\author{Куцелабский Е.С.}
\setsupervisor{Петров П.С.}{}

\begin{document}
	\makereporttitle
	\maketableofcontents
	
	\begin{abstract}
		Целью работы является разработка (текстового редактора?) для open-source движка Citrus. Вставьте текст.
	\end{abstract}

	\section{Введение}
	\subsection{Глоссарий}
	Наверное что-нибудь будет		
	\subsection{Описание предметной области}
	\par Вставить информацию о Game Forest, о движке Citrus. Упомянуть об использовании текста в виджетах в Orange и Tangerine, об использовании текста в играх.	
	\par UPD: А может тут в принципе надо писать о том, что текст должен как-то рисоваться?
	Рассказать о старой текстовой системе и о её недостатках (а это медленная работа при больших объемах данных и общая недостаточная эффективность). Возникающие ограничения: не хватает фпс в играх -> потеря игроков, отказ от поддержки старых устройств -> потеря денег; долгое ожидание обработки текста, замедление консолей Orange и Tangerine при долгой работе над большими проектами, общее замедление движка из-за обилия текстовых виджетов -> падение производительности сотрудников.	
	\par Подходы к решению проблемы: прямо сейчас просто терпим и избегаем больших объемов текста, играем хватает, а художники и так потерпят. Варианты решения: переписать частями наиболее медленные части (это делаю я), переписать вообще всё целиком, не знаю что третье придумать.
	\par Пределы применения: границы класса задач? Ну не знаю, любые текстовые редакторы. Потенциально мой фикс можно использовать где-угодно, однако похоже, что придётся поменять всю визуальную часть, поскольку в данный момент она использует цитрусовый Lime.		
	\subsection{Неформальная постановка задачи}
	В рамках данной работы требуется выполнить следующие задачи:
	\begin{enumerate}
		\item Изучить имеющиеся подходы к реализации текстовых редакторов
		\item Выбрать из имеющихся наиболее подходящий под нужды компании Game Forest или ИЗОБРЕСТИ ВЕЛОСИПЕД
		\item Тщательно протестировать систему, чтобы переход на неё в старых игровых проектах ничего не сломал
		\item Убедиться в том, что случился прирост производительности
	\end{enumerate}
	\subsection{Математические методы}
	А они у меня есть? Вроде бы не особо, только структуры данных	
	\subsection{Обзор существующих методов решения}
	\par Ну тут можно разгуляться. Вставляем табличку, туда вим, саблайм текст, атом, вижак. А ещё вставить текущий текстовый движок хитруса.
	\par Как вставлять таблицу в техе?	
	\par А сюда вывод пихнём
	\subsection{План работ}
	\par Для преддипломной практики нужно будет что-то родить. Для диплома, думаю, не нужно. Наверное стоит вставить план работ по реализации финальной части диплома.
	
	\section{Требования к окружению}
	\subsection{Требования к аппаратному обеспечению}
	\begin{itemize}
		\item Вставить комп, поддерживающий цитрус, это надо спросить у челов.
		\item Вставить смартфоны, поддерживающие цитрус
		\item Для компов, кстати, возможно стоит указать УСТРОЙСТВО ВВОДА
	\end{itemize}
	\subsection{Требования к программному обеспечению}
	\begin{itemize}
		\item Ну тут тоже ОС для хитруса
		\item Версии iOS и Android
	\end{itemize}
	\subsection{Требования к пользователям}
	\begin{itemize}
		\item А мне оно надо?
		\item Ну можно написать о базовых навыках ввода текста на компе или смартфоне
	\end{itemize}
	\subsection{Организационные требования}
	\par Ставлю на то, что их нет, но возможно стоит вставить штуки в духе ПОДНЯТЬ ВСЕХ ТЕСТИРОВЩИКОВ НА ВЫЯВЛЕНИЕ БАГОВ
	
	\section{Архитектура системы}
	\par Пишем много и подробно. У меня будет сама структура данных, subEditor, обеспечивающий к ней доступ, парсилка текста, рендерер, инпут контроллер
	
	\section{Спецификация данных}
	\subsection{Описание формата или структуры данных}
	\par Это, наверное, про PieceTree
	\subsection{Описание сущности}
	\par А оно у меня есть?
	\subsection{Описание протокола}
	\par Ну тут уж наверняка мимо
	
	\section{Функциональные требования}
	\par Тут надо понаписать много-много всего
	\subsection{Библиотека подпрограмм(классов)
	\par Самый главный раздел в мире, описываем все классы мира, а их у меня достаточно.
	\par Еще всякие схемы данных и прочее
	
\end{document}