\documentclass{fefu}

\begin{document}
	\setschool{ШКОЛА ЕСТЕСТВЕННЫХ НАУК}
	\setdepartment{кафедра информатики, математического и компьютерного моделирования}{А.Ю.Чеботарев}
	\setgroup{Б8403а}
	\title{о прохождении преддипломной практики\\направление подготовки 01.03.02 Прикладная математика и информатика\\профиль «Системное программирование»}
	\setdates{29}{апреля}{2019}{22}{июня}{2019}
	\setweeks{8}
	\setplace{кафедре информатики, \\математического и компьютерного \\моделирования}
	\author{Куцелабский Е.С.}
	\setsupervisor{Петров П.С.}
	
	\makereporttitle
	\tableofcontents
	\newpage
	
	\begin{abstract}
		\par Целью работы является разработка текстового редактора для 
		open-source движка Citrus \cite{CitrusRepo}. 
		В работе изложены особенности реализации и принятые решения.
		Выполнено сравнение производительности между предыдущей версией редактора и
		теперешней (\textbf{не очень слово}) реализацией.
	\end{abstract}

	\section{Введение}
		\subsection{Глоссарий}
			\textbf{После написания отчёта вставить}		
		\subsection{Описание предметной области}
			\textbf{Вставить информацию о Game Forest, о движке Citrus.}
			\par Упомянуть об использовании текста в виджетах в Orange и Tangerine, об 
			использовании текста в играх.	
			\par UPD: А может тут в принципе надо писать о том, что текст должен как-то 
			рисоваться?
			Рассказать о старой текстовой системе и о её недостатках (а это медленная работа 
			при больших объемах данных и общая недостаточная эффективность). Возникающие 
			ограничения: не хватает FPS в играх -> потеря игроков, отказ от поддержки старых 
			устройств -> потеря денег; долгое ожидание обработки текста, замедление консолей 
			Orange и Tangerine при долгой работе над большими проектами, общее замедление 
			движка из-за обилия текстовых виджетов -> падение производительности сотрудников.	
			\par Подходы к решению проблемы: прямо сейчас просто терпим и избегаем больших 
			объемов текста, играм хватает, а художники и так потерпят. Варианты решения: 
			переписать частями наиболее медленные части (это делаю я), переписать вообще всё 
			целиком, не знаю что третье придумать.
			\par Пределы применения: границы класса задач? Ну не знаю, любые текстовые 
			редакторы. Потенциально мой фикс можно использовать где-угодно, однако похоже, 
			что придётся поменять всю визуальную часть, поскольку в данный момент она 
			использует цитрусовый Lime.		
		\subsection{Неформальная постановка задачи}
			В рамках данной работы требуется выполнить следующие задачи:
			\begin{enumerate}
				\item Изучить имеющиеся подходы к реализации текстовых редакторов
				\item Учитывая особенности рабочих процессов в компании Game Forest выбрать 
				наиболее подходящую реализацию или предложить свою
				\item Провести тестирование системы в целях избегания ошибок, поскольку их цена 
				чрезвычайно высока \textbf{звучит глупо}
				\item Исследовать изменение производительности
			\end{enumerate}
			\textbf{Цитрус open-source, так нужно ли писать о политике распространения?}
		\subsection{Математические методы}
			\textbf{Опускаем раздел, все структуры данных в разделе 8}	
		\subsection{Обзор существующих методов решения}
			\par \textbf{VS, VS Code, Atom, vim, Notepad, Sublime text, хорошо взять из статьи,
			так описаны алгоритмы и структуры данных}.
			\textbf{Я не выбираю конкретный редактор, а смотрю на их варианты реализации фич, 
			потому-что заказчик сказал переделать именно редактор в цитрусе.}
			\begin{center}
				\begin{tabular}{ |l|c|r|}
					\hline
					a & b & c \\
					\hline
				\end{tabular}
			\end{center}
			\par А сюда вывод пихнём
		\subsection{План работ}
			\par \textbf{ТОЛЬКО ДЛЯ ПРЕДДИПЛОМНОЙ ПРАКТИКИ}
			\begin{itemize}
				\item Изучение подходов к реализации текстовых редакторов (1-2 недели)
				\item Изучение кодовой базы движка Citrus в целях нахождения места для внесения
				изменений (1 неделя)
				\item Реализация текстового редактора (4 недели) \textbf{Какая-то тавтология 
				да и звучит сомнительно}
				\item Подготовка отчёта (1 неделя) \textbf{Это вообще в план должно входить?}
			\end{itemize}
			\textbf{Суммарно набралось 8 месяцев - длительность практики.}
	\section{Требования к окружению}
		\subsection{Требования к аппаратному обеспечению}
			\begin{itemize}
				\item \textbf{Компьютер, способный осилить цитрус и смартфон, способный 
				потянуть сделанные на нём игры}
				\item \textbf{Возможно стоит указать УСТРОЙСТВО ВВОДА (клавиатура, сенсорный
				экран)}
			\end{itemize}
		\subsection{Требования к программному обеспечению}
			\begin{itemize}
				\item ОС Windows 8.1 или выше; MacOS \textbf{Уточнить версию}; дистрибутивы 
				Linux, поддерживающие среду Mono
				\item Android 4 или выше \textbf{уточнить}; iOS \textbf{узнать версию};
			\end{itemize}
			\textbf{А может я писал библиотеку? Тогда надо писать про всякие .Net}
		\subsection{Требования к пользователям}
			\textbf{Кажется, у меня их нет}
		\subsection{Организационные требования}
			\textbf{Опускаем раздел}
	\section{Архитектура системы}
		\par \textbf{ТУТ НАДО ПИСАТЬ ПРО ВИДИМОЕ ДЛЯ ПОЛЬЗОВАТЕЛЕЙ, А ВИДИМЫЕ ТОЛЬКО 
		КОМОНТЕКСТ И ЭДИТОР. Пожалуй, весь этот текст надо перенести в раздел 8 "Модули и
		алгоритмы}
		\par Система состоит из следующих модулей \textbf{А может не надо конкретных названий
		и стоит написать в общих чертах? Нужно ли писать про классы, которые я почти не 
		трогал (например, про Renderer и TextParser)?}:
		
		\begin{itemize}
			\item PieceTree - структура данных \textbf{Вообще-то класс} для обработки 
			внутреннего представления текста
			\item SubEditor - класс, обеспечивающий доступ к PieceTree и служащий для 
			обработки вызовов высокого уровня \textbf{Это так пишется?} (например, 
			"Переместить курсор", "Вставить символ")
			\item TextParser - класс, назначение которого - выделить набор фрагментов -- слов, 
			пробелов, переносов строки -- из исходного текста
			\item TextRenderer - класс, в котором слова разделяются по строкам с учетом 
			максимальных размеров виджета, а затем каждой строки вычисляется её позиция 
			в пикселях
			\item Renderer - класс, выполняющий отрисовку текста
			\item CommonText - связующее звено между модулями
			\item Editor - класс, обеспечивающий обработку пользовательского ввода (и 
			передачу соответствующих команд низкоуровневым модулям)
		\end{itemize}
	\section{Спецификация данных}
		\subsection{Описание формата или структуры данных}
			\textbf{Тут стоит написать про тэги в тексте, да?}
		\subsection{Описание сущности}
			\textbf{Опускаем раздел}
		\subsection{Описание протокола}
			\textbf{Опускаем раздел}
	\section{Функциональные требования}
		\par Тут надо понаписать много-много всего
		\subsection{Библиотека подпрограмм(классов)}
			\par Самый главный раздел в мире, описываем все классы мира, а их у меня 
			достаточно.
			\par Еще всякие схемы данных и прочее
	\section{Требования к интерфейсу}
		\textbf{С точки зрения UI стоит написать про Editor и CommonText, а именно про 
		шрифты, поддержку переносов, переносы слов и всё такое. С точки зрения API надо писать
		про вещи в духе "Вставить текст" или "Отменить выделение". А нужно ли писать про API?}
	\section{Прочие требования}
		\subsection{Требования к надежности}
			\par В случае возникновения ошибок, возможны падения приложений у игроков. Это 
			отрицательным образом сказывается на их вовлеченности в игровой процесс, что
			приводит к падению экономических показателей.
			\par Для предотвращения подобных ситуаций стоит провести тщательное тестирование
			программного кода на предмет ошибок (на низком уровне - с помощью 
			юнит-тестирования, на высоком - силами штатных тестировщиков).
		\subsection{Требования к безопасности}
			\textbf{Опускаем раздел или пишем про текстовые поля для ввода паролей}
		\subsection{Требования к производительности}
			Вся цепочка операций (от обработки изменений текста до 
			отрисовки) должна выполняться не более чем за 30 миллисекунд (для отрисовки 30 
			кадров в секунду).
	\section{Проект}
		\subsection{Средства реализации}
			Все части системы были реализованы на языке C\# 7.0, 
			версия .NET Framework 4.7.1, поскольку это стандарт для компании Game Forest.
		\subsection{Структуры данных}
			\textbf{Рассказать про PieceTree, нарисовать схемы}
		\subsection{Модули и алгоритмы}
			\textbf{Переносим из раздела выше и добавляем подробностей}
		\subsection{Проблемы и решения}
			\par В ходе разработки пришлось столкнуться со следующими \textbf{дописать 
			подводку}
			\begin{itemize}
				\item Предыдущий код TextRenderer трудно читается из-за функций размером в
				200 строк. Рефакторинг значительно улучшил читаемость \textbf{А мои функции 
				усложнили ха-ха}
				\item \textbf{Что-то там еще было}
			\end{itemize}
		\subsection{Стандарт кодирования}
			\textbf{Пишем стандарт Game Forest}
		\subsection{Проект интерфейса}
			\textbf{Вставить снимок CommonText что-ли}
	\section{Реализация и тестирование}
		\textbf{Всё по плану}
		\subsection{Вычислительный эксперимент}
			\textbf{Тут вставить обещанное исследование/сравнение производительности}
	\section*{Заключение}
		\textbf{Всё по плану}
	\newpage
	\bibliographystyle{ugost2008ls}
	\bibliography{references}
	
\end{document}